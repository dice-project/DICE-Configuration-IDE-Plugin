\newpage
\subsection{Integration: Eclipse, Jenkins, BO4CO, Testbed}
This section details the implementation to provide a fully integrated development environment that triggers configuration optimisation on remote automation server and runs tests on remote testbed. The following deliverables are covered:
\begin{itemize}
\item Integrate Eclipse and Jenkins for triggering parameterised builds with configuration files remotely
\item Integrate Eclipse and Jenkins to retrieve and display BO4CO configuration results
\item Integrate BO4CO tool and remote Storm testbed to deploy tests with different configuration parameters and retrieve performance metrics
\item Integrate Jenkins automation server and MATLAB-based BO4CO tool to start experiment
\end{itemize}

\subsubsection{Eclipse and Jenkins Integration}
The aim is to provide integration between the Eclipse plugin described in the above sections, and a remote Jenkins server containing the Configuration optimisation tool instance. The tool requires an experiment configuration file to execute, and hence there is a need to remotely trigger Jenkins' parameterised build with a file parameter.\\
A Jenkins \textit{project} has to be created, with remote triggering enabled. This opens the option to specify a token used for added security when remotely executing the build from the Jenkins API. The parameterised trigger option has to be selected, with a file parameter specified.\\
There is an existing Eclipse plugin that is capable of running builds and monitoring the status of remote Jenkins server, called the Hudson/Jenkins Mylyn Builds Connector \cite{mylyn}. Howver, this plugin does not support triggering parameterised builds and does not support sending file (or any) parameters. The possibility to re-use code from the plugin was investigated, but it was found that the code required to trigger parameterised build significantly differed from the existing code that only supported simple builds. Therefore the integration was build from scratch.\\
Jenkins offers an API that opens functionality to remote access. The API requires three security components to remotely trigger builds: \cite{remoteapi}
\begin{itemize}
\item User-defined project remote trigger token\\
The project token is a simple plain text token that is sent as one of the arguments in the HTTP request URL string.
\item Credentials of an authorised user - username and password\\
Pre-emptive authentication is used to authenticate the user with provided username and password. The pre-emptive authentication for Apache HTTP client is done by implementing a HTTPRequestInterceptor that intercepts all HTTP requests and injects the authentication component.
\item CSRF\\
There is no available implementation of the CSRF authentication for Jenkins and Apache HTTP client, possibly due to the fact that CSRF was only enforced recently for new installations of Jenkins 2.x upwards.\\
The CSRF crumb can be obtained by requesting the Jenkins API's CrumbIssuer, and the responds message is parsed to extract the crumb value. It is stored and used in subsequent requests.
\end{itemize}
Remotely triggering parameterised builds is documented in the Jenkins remote access API. \cite{remoteapi} Text based parameters such as Strings and integers were simple appended onto the trigger HTML request URL as parameters. However there was no documentation on how to include a file parameter \cite{pbuild}, although there was a specific example using cURL \cite{remoteapi}. Unfortunately the general requirement is unclear, and there are multiple ways of submitting a file as an HTML form using Apache HTTP client. Due to the lack of documentation, it was down to trial and error to find out which type of file submission did Jenkins support.\\
The file parameter is attached to the HTTP request as an HTTP entity created by a MultipartEntityBuilder, with the TextBody specify the plaintext file name in JSON, and file content attached as BinaryBody. As far as the trial and error results show, this exact approach is the only solution to triggering a Jenkins parameterised build remotely with a file parameter from a Java / Apache based HTTP request.\\
To retrieve the result from Jenkins Server to Eclipse plugin when the ``Show results'' button is pressed, an HTTP request is sent. Jenkins API provides a plaintext HTTP reponse that contains the entire console log output of the last successful build. The configuration optimisation results are extracted from this output.

\newpage
\subsubsection{Bug Fixes and Improvements to BO4CO}
The Configuration optimisation tool (BO4CO) requires integration with the following services that it relies upon to function:
\begin{itemize}
\item DICE Jenkins Continuous Integration service - for monitoring experiments
\item DICE Deployment Service - for deploying experiments
\item DICE Monitoring Platform - to monitor performance of experiments
\item Storm cluster - for executing experiments
\item Zookeeper cluster - for deploying experiments
\item Kafka cluster - for deploying experiments
\item Hadoop cluster - for executing experiments
\end{itemize}
The integration with these services was designed to be included in the BO4CO code. However, upon testing the tool would not function properly. The reasons for malfunction is mainly due to incorrect initialisation, and outdated set up in connection with services, as detailed below:
\begin{itemize}
\item Hard coded reading of services information from experiment configuration file - The BO4CO code is hard coded to read the sequence of services in a fixed order in \textit{init.m}, as listed here: 
	\begin{enumerate}
	\item Continuous Integration server
    \item DICE Deployment service
    \item DICE Monitoring service
    \item Storm cluster
    \item Kafka
    \item Zookeeper
    \item Hadoop cluster
	\end{enumerate}
In addition the the ordering problem, it also requires that all of the above services be present. That would be a rare case and seems to be a flaw, because a big data application would not typically use Storm and Hadoop frameworks at the same time.\\
The issue was resolved by fixing the ordering of services in the Eclipse plugin, and editing the BO4CO code to remove requirement of multiple big data frameworks.
\item Incorrect access to non-existent field in HTTP response from the DICE Deployment service - The tool requests information from the DICE Deployment Service about the experimental Nimbus cluster, and attempts to access the \textit{storm\_nimbus\_address.value} field from the HTTP response in \textit{get\_container\_details.m}. It appears that the DICE Deployment Service's API has been updated, and the required information is now in the HTTP response field \textit{storm\_nimbus\_host.value}.
\item Erroneous calls to set up SSH connection - The tool attempts to set up SSH connections to services using the Ganymed SSH-2 library, in \textit{setup.m}. The code calls \textit{ssh2\_setup()} to initialise the settings three times, which causes the first two connections settings to be overwritten and hence become unnecessary. The argument provided to the third call is also unsupported by the library. It causes the BO4CO code to run with a warning message that also appears in the currently released version of the tool. Upon investigation we were unable to infer the reason and possible thinking behind adding this argument to the method call. The solution is to remove the related code, so that the connection setup is only called once with the correct arguments (no arguments).
\item Invalid selection of parameter options / Invalid array access - Following latin hypercube design, the tool attempts to run experiment on the parameter option at the generated index positions. However, the Latin hypercube design in \textit{lhsdesign4grid.m} generates non-integer indices, causing the array access to be invalid. The solution is to round these indices to run experiment on the nearest parameter option.
\end{itemize}

\newpage
\subsubsection{Extension on BO4CO}
The Big Data Auto-tuning tool aims to automatically improve big data application performance by configuration optimisation. The configuration parameters of big data frameworks fall into these four categories:
\begin{itemize}
\item Integer - the parameter may take any integer value between a set of lower and upper bounds.
\item Percentage - the parameter may take any value between 0 and 1.
\item Boolean - the parameter value may be \textsc{true} or \textsc{false}
\item Categorical - the parameter may take any value from a list of Strings options
\end{itemize}
In the originally release version, BO4CO tool only supported numerical parameter values, i.e. Integer and Percentage parameters. In order to have ``fully automatic'' configuration optimisation, all parameters should be supported. Hence we extend the BO4CO tool to support Categorical and Boolean types.\\
The restriction on the existing implementation of BO4CO is an implementation restriction - an issue with data typing throughout the code. The main problem is not cause by the logical differences between numerical (continuous data) and categorical (discrete data). BO4CO treats numerical data as if it were discrete data, because it only considers the values provided in the \textit{options} list. The code does not attempt to interpolate or extrapolate numerical values from the parameter options, nor does it attempt to apply any mathematical operation to the parameter options. In fact, one can argue that the core BO4CO algorithm is entirely oblivious to the value of the parameter option, and works strictly by selecting an index (that points to a parameter value), instead of directly selecting any values.\\
The parts of the tool which restricted data types of parameters include:
\begin{itemize}
\item Processing input - existing implementation parsed all parameter values into numerical vectors
\item Initialisation - existing implementation used formatted strings requiring numerical values as arguments
\item Connecting and deploying to Storm cluster - existing implementation of command string concatenation did not support data structures
\item Data aggregation, logging and generating output - formatted output files required numerical arguments / arguments all of identical data type in a matrix
\end{itemize}
In order to maintain the format of output files for compatibility issues with other tools depending on the BO4CO configuration optimisation tool, the output stage required all parameter values to be numerical. This has been resolved by reporting the option's index number instead of its actual string value for categorical and boolean parameters. Other parts of the BO4CO code can be safely modified to support both data types, and uses the string value of the option when processing categorical and boolean parameters. The following steps were taken to implement this extension:
\begin{itemize}
\item Processing input - Input data parsed as vector of String using \textit{strsplit} in \textit{domain2option.m} instead of \textit{str2num}.
\item Initialisation - String generation using \textit{strjoin} in \textit{f\_storm.m} instead of \textit{fprintf} to remove formatting requirement
\item Connecting and deploying to Storm cluster - Command generation using \textit{strcat} and \textit{char(cmd)} to create string, instead of primitive string concatenation using the \textit{[ ]} operator.
\item Data aggregation, logging and generating output -  represent categorical and boolean options with their index value instead of string value, to maintain formatted output files that required numerical arguments / arguments all of identical data type in a matrix. If the string value is required, it can be obtained by checking the original experiment configuration file for list of options, and selecting the string value corresponding to the index.
\end{itemize}

\newpage
\subsubsection{Jenkins and BO4CO MATLAB Integration}
Integrating the BO4CO tool to Jenkins is the final integration step that completes the fully automated Big Data Auto-tuning tool.\\
As mentioned in the above section, a Jenkins project is created with file parameter from the Eclipse plugin. By executing shell script or Windows batch commands on the Jenkins project, the file is moved into the working directory where the BO4CO code is located. The BO4CO tool can be run in deployed mode as a compiled MATLAB executable, and results are retrieved by access the output files at pre-defined location. To allow Eclipse plugin to retrieve the results, the file containing configuration optimisation results is printed on the console which is logged by the Jenkins API.\\
It is also possible to configure Jenkins to run with the BO4CO source code. This requires a MATLAB installation on the Jenkins server. To automate execution of BO4CO source code with Jenkins, the following arguments are required for the \textit{matlab} command:
\begin{itemize}
\item \textit{-nodisplay -nosplash -nodesktop} - To ensure that graphic elements are not displayed and a MATLAB editor is not started.
\item \textit{-r ``main;exit''} - To run \textit{main.m} and then exit MATLAB.
\item \textit{-wait} - To return only after the BO4CO experiments finish, instead of terminating when MATLAB has been successfully started.
\item \textit{-sd ``source-code-directory''} - Provides path to source code directory to use as workspace.
\end{itemize}
Developers attempting to set up with Jenkins Servers on a Windows based system will have to take extra step. An unofficial guide exists for integrating MATLAB with Jenkins, which assumes the use of a Linux based system. There is no documentation on integrating MATLAB with a Windows based Jenkins installation. By investigation, it was found that the following two extra steps must be taken by developers that choose to use this set up:
\begin{itemize}
\item Jenkins must be installed as a standalone process. Jenkins for Windows is provided as an installer that installs as a system service. This creates an issue because the system processes cannot execute MATLAB. The solution is to run Jenkins from the command line with the user's access rights. Jenkins can be run as a standalone Java package, or wrapped in an off-the-shelf server instance such as Apache Tomcat.
\item An extra argument is required to the command, that points to the MATLAB license file - \textit{-c ``path-to-license-file''}
\end{itemize}
